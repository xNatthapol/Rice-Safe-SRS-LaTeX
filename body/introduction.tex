\chapter{Introduction}
\label{chap:introduction}

\section{Background}
\label{section:background}

Rice is a globally vital staple crop, serving as a primary food source and a key agricultural commodity. According to the United States Department of Agriculture (USDA), global rice production for the 2024/2025 period is estimated at approximately 532.7 million metric tons. However, rice cultivation faces increasing threats from diseases caused by pathogens such as fungi, bacteria, and viruses, which significantly impact crop yield and quality. For instance, rice blast disease alone can cause annual yield 
reductions of about 10-30\%, leading to substantial financial losses for farmers.

In rural and remote regions, farmers often lack access to reliable tools for detecting and managing rice diseases, leading to delays and inconsistent treatment. This lack of effective disease management worsens the spread and impact of these diseases. Additionally, many farmers lack the necessary knowledge to effectively identify and address these diseases. There is an urgent need for more accessible and effective solutions to support farmers in tackling rice diseases.

\section{Problem Statement}
\label{section:problem-statement}

Rice farmers often face challenges due to the lack of reliable tools for detecting and managing rice diseases in a timely and effective manner. Many farmers seek advice from experts, but these approaches can be inconsistent and difficult to access. As a result, delays in identifying and addressing diseases can lead to increased crop losses and threaten farmers' livelihoods.

Additionally, there is no centralized platform for farmers to share information on rice diseases and collaborate with each other. The absence of a structured knowledge repository limits their ability to respond quickly to disease outbreaks and hinders efforts to control the spread of infections. This lack of collaboration and timely information exchange leads to challenges in coordinating efforts and sharing timely updates about disease threats.

\section{Solution Overview}
\label{section:solution-overview}

RiceSafe is a mobile app designed to help farmers diagnose rice plant diseases quickly and accurately. By combining farmers’ descriptions of symptoms with images of affected plants, RiceSafe uses AI technologies like MSC-TextCNN, MSC-ResViT, and CT-CNN to improve diagnosis. The app offers treatment suggestions, including remedies, and sends alerts to nearby farmers about potential disease outbreaks. It also includes a community hub where farmers can share their experiences, disease detection tips, and treatment methods. With a knowledge base on rice diseases, their symptoms, causes, and treatments, RiceSafe aims to provide an easy-to-use, accessible solution that helps farmers detect diseases early, reduce crop losses, and promote sustainable farming.

\subsection{Features}
\label{subsection:features}

\begin{enumerate}[leftmargin=80pt]
    \item Rice Disease Detection: Uses a multimodal AI model to analyze text and image inputs for accurate disease identification.
    \item Treatment Recommendations: Offers remedies and practical ways to manage the diagnosed disease.
    \item Disease Library: Provides comprehensive details about rice diseases, their symptoms, causes, and treatments.
    \item Outbreak Alerts: Notifies farmers in the vicinity of potential disease outbreaks to prevent further spread.
    \item Community Hub: Enables farmers to share disease detection insights, experiences, and treatment methods.
\end{enumerate}

\section{Target User}
\label{section:target-user}

RiceSafe is designed specifically for small-scale farmers, who often lack access to expert agricultural guidance and rely on personal experience to identify rice diseases. This traditional approach can lead to delays in diagnosis and treatment, increasing crop losses.

With RiceSafe, farmers can quickly and accurately detect rice diseases using AI-powered analysis of text and images. The app provides treatment recommendations and outbreak alerts, helping farmers take timely action to protect their crops. By making disease management more accessible and reliable, RiceSafe empowers small-scale farmers to improve yields and ensure food security.


\section{Benefit}
\label{section:benefit}

Farmers can benefit by using RiceSafe, as it enhances disease diagnosis using AI analysis techniques to detect rice diseases and recommend effective treatments promptly to reduce crop damage caused by diseases.

RiceSafe goes beyond managing diseases by promoting community involvement among farmers to exchange disease findings and advice on treatments together. The app's collaborative aspect enriches knowledge sharing within the farming community. Plays a role in preventing diseases through early detection and outbreak alerts to control their spread effectively. RiceSafe ultimately promotes farming practices by advocating for friendly treatments and enhancing agricultural techniques for better results.


\section{Terminology}
\label{section:terminology}

\begin{enumerate}
    \item Multimodal AI Model: An artificial intelligence model that integrates multiple forms of input (e.g., text descriptions and images) to improve diagnosis and analysis.
    \item MSC-TextCNN: A deep learning model used for analyzing text data (in this case, symptom descriptions) for disease diagnosis.
    \item MSC-ResViT: A model used in the RiceSafe app for image recognition and analysis to help detect diseases in rice plants.
    \item CT-CNN: A deep learning model that combines convolutional neural networks with other techniques to improve disease detection from images.
\end{enumerate}