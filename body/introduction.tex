\chapter{Introduction}
\label{chap:introduction}

\section{Background}
\label{section:background}

Rice is a vital staple crop, especially in Asia, where it serves as a primary food source and a key agricultural commodity. For millions of farmers, rice cultivation is central to their livelihoods and cultural identity. However, the crop faces increasing threats from diseases caused by pathogens such as fungi, bacteria, and viruses, which significantly impact crop yield and quality. The severity of these diseases can lead to major financial losses, affecting not only farmers’ incomes but also food security on a larger scale.

In rural and remote regions, farmers often lack access to expert resources and are forced to rely on personal experience to identify and manage these diseases. This method can be unreliable and lead to delays in diagnosis and treatment. The situation is further complicated by climate change and the rise of new, evolving pathogens, making it increasingly difficult for farmers to cope with the growing challenge of disease management. There is an urgent need for more accessible and effective solutions to support farmers in tackling rice diseases.

\section{Problem Statement}
\label{section:problem-statement}

Rice farmers in areas often struggle with the challenge of not having access to reliable tools for detecting and treating rice diseases promptly and effectively. The majority of farmers rely on their experiences. Seek advice from experts, but these methods can be unreliable or difficult to obtain. This leads to delays in identifying and treating diseases in crops, which ultimately increases losses and poses a threat to the security of farmers.

Additionally, farmers currently lack a platform to exchange information on diseases and work together in time. The absence of a knowledge repository hampers their ability to quickly address disease outbreaks and impairs endeavors to curb disease transmission. Resources and communication methods also undermine the efficiency of disease control, highlighting the necessity for a solution that allows for precise and prompt diagnoses as well as seamless information sharing among farmers to minimize the effects of rice diseases.

\section{Solution Overview}
\label{section:solution-overview}

RiceSafe is a mobile app designed to help farmers diagnose rice plant diseases quickly and accurately. By combining farmers’ descriptions of symptoms with images of affected plants, RiceSafe uses advanced AI technologies like MSC-TextCNN, MSC-ResViT, and CT-CNN to improve diagnosis. The app offers personalized treatment suggestions, including affordable, natural remedies, and sends alerts to nearby farmers about potential disease outbreaks. It also includes a community hub where farmers can share their experiences, disease detection tips, and treatment methods. With a comprehensive knowledge base on rice diseases, their symptoms, causes, and treatments, RiceSafe aims to provide an easy-to-use, accessible solution that helps farmers detect diseases early, reduce crop losses, and promote sustainable farming.

\subsection{Features}
\label{subsection:features}

\begin{enumerate}[leftmargin=80pt]
    \item Rice Disease Detection: Uses a multimodal AI model to analyze text and image inputs for accurate disease identification.
    \item Treatment Recommendations: Suggests cost-effective and natural remedies tailored to the diagnosed disease.
    \item Outbreak Alerts: Notifies farmers in the vicinity of potential disease outbreaks to prevent further spread.
    \item Community Hub: Enables farmers to share disease detection insights, experiences, and treatment methods.
    \item Disease Library: Provides comprehensive details about rice diseases, their symptoms, causes, and treatments.
\end{enumerate}

\section{Target User}
\label{section:target-user}

RiceSafe is designed specifically for small-scale farmers, who often lack access to expert agricultural guidance and rely on personal experience to identify rice diseases. This traditional approach can lead to delays in diagnosis and treatment, increasing crop losses.

With RiceSafe, farmers can quickly and accurately detect rice diseases using AI-powered analysis of text and images. The app provides treatment recommendations and outbreak alerts, helping farmers take timely action to protect their crops. By making disease management more accessible and reliable, RiceSafe empowers small-scale farmers to improve yields and ensure food security.


\section{Benefit}
\label{section:benefit}

Farmers can benefit greatly from using RiceSafe, as it enhances disease diagnosis using AI analysis techniques to detect rice diseases and recommend effective treatments promptly to reduce crop damage caused by diseases.

RiceSafe goes beyond managing diseases by promoting community involvement among farmers to exchange disease findings and advice on treatments together. The app's collaborative aspect enriches knowledge sharing within the farming community. Plays a role in preventing diseases through early detection and outbreak alerts to control their spread effectively. RiceSafe ultimately promotes farming practices by advocating for friendly treatments and enhancing agricultural techniques for better results.


\section{Terminology}
\label{section:terminology}

\begin{enumerate}
    \item Multimodal AI Model: An artificial intelligence model that integrates multiple forms of input (e.g., text descriptions and images) to improve diagnosis and analysis.
    \item MSC-TextCNN: A deep learning model used for analyzing text data (in this case, symptom descriptions) for disease diagnosis.
    \item MSC-ResViT: A model used in the RiceSafe app for image recognition and analysis to help detect diseases in rice plants.
    \item CT-CNN: A deep learning model that combines convolutional neural networks with other techniques to improve disease detection from images.
\end{enumerate}